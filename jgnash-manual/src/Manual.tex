\documentclass[letterpaper,12pt]{book}
\usepackage[titletoc]{appendix}
\usepackage[margin=.75in,heightrounded]{geometry}
\usepackage{fancyhdr}

\usepackage[T1]{fontenc}
\usepackage[utf8]{inputenc}
%\usepackage{lmodern}					%font
\usepackage{fourier}                    %font

\usepackage{tabularx}
\usepackage{graphicx}
\usepackage{tikz}
\usepackage{awesomebox}
\usepackage[framemethod=TikZ]{mdframed}
\usepackage{menukeys}
\usepackage{float}
\usepackage{listings}
\usepackage{enumitem}
\usepackage{needspace}
\usepackage{xcolor}
\usepackage{textcomp}
\usepackage{tocloft}
%\usepackage{fontawesome}
\usepackage[singlelinecheck=false]{caption}
\usepackage[hidelinks, linktocpage=true]{hyperref}

\pagestyle{fancy}

\floatstyle{plaintop}
\restylefloat{table}                    % control table placement

\hyphenation{JGNASH}                    % prevent hyphenation
\hyphenation{jGnash}

\parindent0pt  \parskip10pt            % make block paragraphs

\setlength\headheight{16pt}
\setlist[description]{style=nextline}    % newline for descriptive lists

\setlength{\aweboxrulewidth}{0.8pt}    % reduce width the awesome box vertical line
\setlength{\aweboxcontentwidth}{1.0\linewidth}

\renewcommand{\arraystretch}{1.5}        % row height of tables

\renewmenumacro{\directory}[>]{pathswithfolder}     % change icon

\renewcommand\labelitemi{$\bullet$}
\renewcommand\labelitemii{$\circ$}
\renewcommand\labelitemiii{$\bullet$}

\hypersetup{
colorlinks,
linkcolor={blue!80!black},
citecolor={blue!80!black},
urlcolor={blue!80!black}
}

\definecolor{light-gray}{gray}{0.98}    % define light gray backgrounds

\mdfdefinestyle{info}{                    % gray boxes
linecolor=gray,
outerlinewidth=0.25pt,
roundcorner=0pt,
innertopmargin=\baselineskip,
innerbottommargin=\baselineskip,
innerrightmargin=10pt,
innerleftmargin=10pt,
backgroundcolor=light-gray
}

\lstset{%    
backgroundcolor=\color{light-gray},
frame=single,
framesep=1em,
aboveskip=0pt,
stringstyle=\ttfamily,
showstringspaces = false,
basicstyle=\scriptsize\ttfamily,
commentstyle=\color{gray},
keywordstyle=\bfseries,
ndkeywordstyle=\bfseries,
identifierstyle=\ttfamily,
numbers=left,
numbersep=14pt,
numberstyle=\tiny,
numberfirstline = false,
breaklines=true,
columns=fixed,
keepspaces=true,
frameround=ffff
}

\lstdefinelanguage{JavaScript}{
keywords={typeof, new, true, false, catch, function, return, null, catch, switch, var, const, let, async, await, if, in, while, do, else, case, break, from},
ndkeywords={class, export, boolean, throw, implements, import, this},
sensitive=false,
comment=[l]{//},
morecomment=[s]{/*}{*/},
morestring=[b]',
morestring=[b]"
}

\pagenumbering{roman}

\title{\textbf{jGnash 3.5.x Manual} }
\author{Craig Cavanaugh}
\date{\today}

\begin{document}

    \begin{titlepage}

        \raggedleft % Right align the title page

        {\color{gray} \rule{4pt}{\textheight}} % Vertical line
        \hspace{0.08\textwidth} % White space between the vertical line and title page text
        \parbox[b]{0.75\textwidth}{ % Paragraph box for holding the title page text, adjust the width to move the title page left or right on the page

        {\Huge\bfseries jGnash 3.5.x Manual}\\[2\baselineskip] % Title
        {\Large\textsc{craig cavanaugh}}

        \vspace{0.5\textheight} % White space between the title block and the publisher

        {\noindent \today}\\[\baselineskip] % Publish date
        }

    \end{titlepage}

    \tableofcontents
    
    %%%%%%%%%%%%%%%%%%%%%%%%%%%%%%%%%%%%%%%%%%%%%%%%%%%%%%%%%%%%%%%%%%%%%%%%%%%%%%%%%
    \chapter{Legal}\label{ch:legal}
    {\bfseries jGnash Manual}

    Copyright~\copyright~2001-2020 Craig Cavanaugh

    jGnash comes with \textbf{ABSOLUTELY NO WARRANTY}.

    This program is free software: you can redistribute it and/or modify it under the terms of the GNU General Public
    License as published by the Free Software Foundation, either version 3 of the License, or (at your option) any later version.

    This program is distributed in the hope that it will be useful, but WITHOUT ANY WARRANTY; without even the implied
    warranty of MERCHANTABILITY or FITNESS FOR A PARTICULAR PURPOSE. See the GNU General Public License for more details.

    You should have received a copy of the GNU General Public License along with this program.
    If not, see \href{http://www.gnu.org/licenses/}{http://www.gnu.org/licenses/}

    Permission is granted to copy, distribute and/or modify this document under the terms of the GNU Free Documentation
    License, Version 1.3 or any later version published by the Free Software Foundation; with no Invariant Sections,
    no Front-Cover Texts and no Back-Cover Texts.

    A copy of the license is included in the section entitled ''GNU Free Documentation License''.

    Linux\textregistered~ is the registered trademark of Linus Torvalds in the U.S.\ and other countries.
    Windows is a registered trademark of Microsoft Corporation in the United States and other countries.
    Java\texttrademark~ and OpenJDK\texttrademark~ are registered trademarks of Oracle and/or its affiliates.
    
    
    %%%%%%%%%%%%%%%%%%%%%%%%%%%%%%%%%%%%%%%%%%%%%%%%%%%%%%%%%%%%%%%%%%%%%%%%%%%%%%%%%
    \chapter{Conventions and Typographical Features}\label{ch:conventions-and-typographical-features}
    
    Below are conventions used throughout this manual. With jGnash being a cross platform application, there
    may be slight variations in the user interface behavior depending on your operating system. 
    
    \begin{tabularx}{\linewidth}{|l|X|}
        \hline 
        \textbf{Convention} & \textbf{Usage} \\ 
        \hline 
        \hline 
        \keys{CTRL + C} & User Interface buttons, Keyboard shortcuts and individual keys. The \keys{CTRL} key is used generically for \keys{CTRL} and \keys{\cmd} 
        depending on your operating system. \\ 
        \hline 
        \menu{File > Open} & Menu commands with the separator representing a level within the menu hierarchy.\\
        \hline 
        \directory{directory} & Indicates a directory within your operating system. \\        
        \hline
        \hyperref[ch:conventions-and-typographical-features]{Link} & An internal or external hyperlink. \\
        \hline
        \texttt{Monospace} & Example text	\\
        \hline               
    \end{tabularx}      
    
    \mainmatter
    \pagenumbering{arabic}

    %%%%%%%%%%%%%%%%%%%%%%%%%%%%%%%%%%%%%%%%%%%%%%%%%%%%%%%%%%%%%%%%%%%%%%%%%%%%%%%%%
    \chapter{Introduction}\label{ch:introduction}
    \includegraphics[scale=.6]{images/jgnash-logo-small}

    jGnash is an open source application for personal finances.

    jGnash enables you to record detailed account and transaction information using proven double entry accounting principles.

    You do not have to be an accountant to understand or use jGnash, but jGnash provides options to make new and
    experienced users feel comfortable using it.

    jGnash's mission is personal fiance and it is not tailored for use as a business accounting application.
    jGnash is being used by small businesses and clubs, but if you require business specific features, you may be
    better served looking for a different solution.

    \section{Features}\label{sec:features}
    A brief list of features is below:

    \begin{itemize}
        \item Double Entry Accounting with reconciliation tools.
        \item Budgeting with multiple scenario options and export to spreadsheet capability.
        \item Investment Accounts and automatic import of Stocks, Bond, and Funds price history.
        \item Hierarchical accounts with automatic roll-up of totals and intelligent handling of mixed currencies.
        \item OFX, QFX, mt940, and QIF import with the ability to pre-process using JavaScript.
        \item Reminders and automatic transaction entry and notifications.
        \item Intelligent handling of multiple currencies and exchange rates with automatic online exchange rate updates.
        \item Printable reports with PDF and spreadsheet export capability.
        \item XML, Binary and SQL database file formats.
        \item Operates on most all modern PC operating system \textit{(Java\texttrademark~11 is required)}.
        \item jGnash will utilize multi-core processors.
    \end{itemize}
    
    %%%%%%%%%%%%%%%%%%%%%%%%%%%%%%%%%%%%%%%%%%%%%%%%%%%%%%%%%%%%%%%%%%%%%%%%%%%%%%%%%
    \section{Installation}\label{sec:installation}
    jGnash is not currently distributed with an automatic installation tool.
    You will be required to perform a couple of manual operations that are easily performed for those with a basic
    understanding of how to use a zip file.
    
    Java\texttrademark~11 or newer must be installed on your computer. jGnash has been tested with Java\texttrademark~12
    and Java\texttrademark~13 if you have a need for a newer release.

    \textit{\textbf{Use of an OpenJDK package is recommended over use of Oracle JDK due to licensing requirements}}

    OpenJDK is a free and open-source implementation of the Java Platform that is downloadable from various sources.
    Java installation is a simple matter of downloading the correct version for your operating system and using the
    automated installer.

    \begin{description}
        \item[\href{https://adoptopenjdk.net/index.html?variant=openjdk11&jvmVariant=hotspot}{AdoptOpenJDK}]
        Easy to install and recommended for Windows users. See Section~\ref{sec:wininstall} for specifics. \\
        \texttt{[https://adoptopenjdk.net/index.html?variant=openjdk11\&jvmVariant=hotspot]}
        \item[\href{https://www.azul.com/downloads/zulu/}{Azul OpenJDK 11}]
        A branded release that is easy to install for most users and is free to use. \\
        \texttt{[https://www.azul.com/downloads/zulu/]}        
        \item [\href{https://jdk.java.net/11/}{OpenJDK}]
        Will require manual installation and is free to use. \\
        \texttt{[https://jdk.java.net/11]}
        \item[\href{https://www.oracle.com/technetwork/java/javase/downloads/index.html}{Oracle Java SE 11}]
        Will require manual installation and licensing is required. \\
        \texttt{[https://www.oracle.com/technetwork/java/javase/downloads/index.html]}
    \end{description}

    \notebox{
    If performing a manual installation of Java, The \texttt{JAVA\_HOME} Environment Variable must be set and the
    Java \directory{bin} directory must be in the execution path.
    \newpage
    \bigskip
    If you have multiple versions of Java installed on your computer, The \texttt{JAVA\_HOME} Environment
    Variable must point to Java 11 or newer and the related Java \directory{bin} directory must be the only version
    in the execution path.
    Mixing JVM and JDK versions will confuse the boot loader.
    }

    After Java is installed, you are ready to install jGnash.
    Simply open the zip file and extract the \directory{jGnash} directory and it's complete contents to a directory of
    your choice, and do not alter the files or locations.
    I usually create a directory named \directory{bin} in my home directory and keep
    the \directory{jGnash} directory in it to better organize my computer.
    When upgrading between versions of jGnash, do not unzip to the same location and overwrite the existing files.
    Use a new location or delete the existing files first.
    
    \subsection{Windows Installation} \label{sec:wininstall}
    
    Adopt OpenJDK works very well for jGnash primarily because it offers options to update the correct
    registry keys so that jGnash.exe can find Java.  When installing, make sure the \textbf{Set JAVA\_HOME variable}
    and the \textbf{JavaSoft (Oracle) registry entries} options are selected for proper jGnash operation.
        
    \begin{figure}[h]
        \caption{Adopt OpenJDK Install Options}
        \includegraphics[width=0.8\linewidth]{images/adopt-open-jdk-install}
    \end{figure}
   
   \notebox{
   	If you are a Windows user with restricted write access to the file system, you may encounter a FileNotFound exception with an underlying "Access is denied" error.  
   	This is caused by jGnash not being able to download and save the necessary JavaFX files for proper operation.
   	\newpage
   	\bigskip
   	The solution is to place jGnash into your \textbf{AppData} folder which should ensure write access.  The AppData folder is typically located within the \textbf{C:\textbackslash Users\textbackslash user} folder which is hidden by default.  You can see the folder if you change the setting to show hidden files within Windows File Explorer.
   }
   
   
    %%%%%%%%%%%%%%%%%%%%%%%%%%%%%%%%%%%%%%%%%%%%%%%%%%%%%%%%%%%%%%%%%%%%%%%%%%%%%%%%%
    \section{Starting jGnash}\label{sec:starting-jgnash}

    After the \texttt{jGnash} directory has been extracted from the zip file, you should see several files in the directory.
    Of interest at this point are the Bash script and \texttt{exe} files.

    If you are running on a Windows\texttrademark~ based computer, you can simply double click on the \texttt{jGnash.exe} file to
    start jGnash.

    If you are running on a Un*x or BSD (macOS) based system you can start jGnash from a terminal as shown below using
    the included Bash script.
    You can also create your own application launcher using the Bash script in your desktop environment of choice.

    \begin{mdframed}[style=info]
        \texttt{./jGnash}
    \end{mdframed}

    jGnash has several advanced features such as running as a portable application or using jGnash as a multi-user home
    networked application. These advanced features are accessible via the command line.
    Please see \hyperref[ch:cmdOptions]{Command Line Options} for more details.

    \section{Running for the First Time}\label{sec:running-for-the-first-time}
    A license acceptance screen will be displayed the first time you start jGnash.
    jGnash will not run unless the license is accepted.
    The short of the license agreement is jGnash is a freely available program comprised of other freely available software,
    and should anything bad happen during use, the authors are not libel for any damages.
    The license also details how jGnash may be distributed and used.

    \notebox{
    If the license agreement sounds daunting, take a look at the license agreements of commercially available personal
    finance applications and you will see similar agreements.
    Myself and just about every other person making software available for free or purchase tries their best to ensure the
    software they create works well and as intended.
    Sometimes bugs do creep in and it does not work quite as planned.
    The advantage of free software is you generally have direct access to the authors, and you have a much larger voice in
    helping the application grow and evolve over time.
    }

    \section{Getting Help and Giving Back}\label{sec:getting-help-and-giving-back}
    The intent of this user guide is to get you off to a good start using jGnash.
    Despite my best attempts, there are those who need a little bit of extra help or have a special need or
    circumstance and require the help of others that have already been around the block a few times.

    The best place to start is the jGnash user group hosted at \\
    \href{https://groups.google.com/forum/#!forum/jgnash-user}{jGnash-User} \texttt{[https://groups.google.com/forum/\#!forum/jgnash-user]}.

    The user group contains a well rounded group of individuals who can help answer just about any question.
    As a courtesy to others, I encourage you to search the group prior to asking a question to see if it's already
    been answered.

    If you have found a bug, or have suggestions for improvement, the group page has links to a bug and feature request
    tracker that can be used to log and track your requests.
    The group forum can be used to post a bug or request, but use of the tracker ensures the request is not lost in
    the mix of discussions.

    If you are well versed in use of jGnash and other personal finance applications, you are encouraged to give back a
    little time and contribute your experience to the group and help others.

    The~\nameref{ch:frequently-asked-questions} chapter will help answer a few common questions.
      
    % ======================
    % Getting Started
    % ======================
    \chapter{Getting Started}\label{ch:getting-started}
    The basic elements of jGnash are Accounts, Currencies, Securities, and Transactions.
    Every account is assigned one currency and every transaction is associated with at least one account.
    Investment Accounts and Investment Transactions will be associated with a Security.

    jGnash supports various Account types that can be arranged into a flexible hierarchical structure.
    The Accounts may be arranged by financial institution, by type, or some other structure.
    It's good practice to organize your Income and Expense Accounts into a logic arrangement that allows you to drill
    down into a layer of more detail.

    The balance of an Account will roll up into it's parent Account within the Accounts view as well as in some reports.

    Below is a typical Expense Account arrangement that allows you to differentiate between different type of automotive
    and food related expenses, but roll them up into more macro level expenses.

    \begin{mdframed}[style=info]
        \begin{itemize}
            \item Expense Accounts
            \begin{itemize}
                \item Automobile
                \begin{itemize}
                    \item Fuel
                    \item Insurance
                    \item Service
                \end{itemize}
                \item Food
                \begin{itemize}
                    \item Dining Out
                    \item Groceries
                \end{itemize}
            \end{itemize}
        \end{itemize}
    \end{mdframed}

    The Account structure may be changed and reordered later, but Accounts may not be deleted if they contain transactions.

    Transactions are used to record daily expenditures as well as income from the sale of personal items, investments, or
    paychecks.

    If you are familiar with other personal finance applications, you may notice that jGnash uses income and expense
    \textit{accounts} instead of income and expense \textit{categories}.
    Functionally, there is no difference, other than jGnash allows you see a detailed transaction register of the
    income and expense accounts as easily as you would look at your bank accounts.

    \section{Editing Environment}\label{sec:editing-environment}
    The jGnash editing environment is not much different then in other application with the exception that it provides a
    few shortcuts to speed up entry of transactions.

    jGnash knows and understands just about any known locale and country setting.
    Depending on your settings, the decimal symbol will change accordingly as well as the displayed format of dates.

    Error checking of required form fields is performed real-time.
    As information is entered you will notice buttons such as \keys{Enter} are enabled and disabled based on validity
    of what has been entered.

    \subsection{Date Fields}\label{subsec:dateFields}
    The date field is freely editable and jGnash will make the best attempt at interpreting an invalid entry, but the results
    are indeterminate.
    Clicking on the button to the right of the field will display a calendar dialog where you can select a date as well.

    The field is flexible enough to allow use of multiple keys as the field separators regardless of the current locale.
    This is done to make entry easier on compact keyboards.
    The usable field separator characters are defined in the table below.

    \includegraphics[scale=.8]{images/date-entry}

    Also, dates can be modified using the keyboard shortcuts defined below.

    \begin{table}[H]
        \begin{tabular}{ll}
            \hline
            \multicolumn{1}{|l|}{\textbf{Keys / Mouse}}                            	& \multicolumn{1}{l|}{\textbf{Function}}            \\ \hline \hline
            \multicolumn{1}{|l|}{\keys{{+}} \keys{\arrowkeyup}, Mouse Scroll Up}   	& \multicolumn{1}{l|}{Increase the date by one day} \\ \hline
            \multicolumn{1}{|l|}{\keys{-} \keys{\arrowkeydown}, Mouse Scroll Down} 	& \multicolumn{1}{l|}{Decrease the date by one day} \\ \hline
            \multicolumn{1}{|l|}{\keys{PgUp}, Mouse Thumb Wheel Up}                	& \multicolumn{1}{l|}{Increase the date by one month} \\ \hline
            \multicolumn{1}{|l|}{\keys{PgDn}, Mouse Thumb Wheel Down}              	& \multicolumn{1}{l|}{Decrease the date by one month} \\ \hline
            \multicolumn{1}{|l|}{\keys{t} \keys{T}}                                	& \multicolumn{1}{l|}{Change to today's date} \\ \hline
            \multicolumn{1}{|l|}{\keys{,} \keys{.} \keys{/} \keys{\textbackslash}} 	& \multicolumn{1}{l|}{Valid field separator for all locales} \\ \hline
        \end{tabular}
        \caption{Date Entry Shortcut Keys}
    \end{table}

    \subsection{Number Fields}\label{subsec:numberFields}

    Numerical entry in jGnash is as easy as typing the desired value into the field.
    Decimal separators are handled according to the configured locale.

    \includegraphics[scale=0.8]{images/basic-decimal-entry} \hspace{10pt} \includegraphics[scale=0.8]{images/advanced-decimal-entry}

    You may also enter arithmetic operators and calculate values within the entry field.

    The arithmetic operators that may be used are \keys{(} \keys{)} \keys{{+}} \keys{{-}} \keys{$\star$} \keys{/} \keys{.} \keys{,}~.

    Traditional arithmetic operator precedence is followed for all calculations.

    \section{Creating a New File}\label{sec:creating-a-new-file}

    When you start jGnash for the first time, you will be presented with a very simple screen.
    At the bottom of the screen application messages and background process indicators are shown.

    \subsubsection*{Creating a new file is done in 5 steps using the wizard}
    \begin{enumerate}
        \item Create a new file using the \menu{File > New} command and follow the prompts in the new file wizard.
        When given the choice to select the \hyperref[subsec:fileTypes]{storage type}, leave it as the default value for now.
        A default file name and location will be provided that you can change if desired.
        If the file already exists, you will be warned you are about to overwrite it.
        \item After selecting the \hyperref[subsec:fileTypes]{storage type} and file name, you will be asked to choose
        the default currency.
        The default currency can be changed at a latter time, and if for some reason your currency of choice is not
        available, you can create a custom currency and set it as the default after the file is created.
        \item Next you can choose the currencies that are available for use.
        Currencies may be added and removed as needed at a later time if needed.

        If needed, custom currencies may be added using the \menu{Currencies > Add/Remove} command.
        As locales change, default currency availability will change as Java is updated.
        The typical need for a custom currency is to support legacy accounting information as countries standardize on the Euro.
        \item After choosing the available currencies, default accounts can be selected if desired.
        If you are new to personal finance software, the defaults will be a good starting point.
        The accounts structure can be easily changed after creating the file to accommodate your own personal needs.
        \item The last step is the Summary page of the wizard.
        Verify everything is to your liking and click on the \keys{Finish} button to create your new file.
    \end{enumerate}

    After the file is created, you are now ready to change, add, or remove accounts as need and begin entering transactions.

    \begin{mdframed}[style=info]
        Encryption and password protection options do not exist in jGnash with the exception of a clear text password
        for client/server operation.

        Encryption is difficult when it involves exporting and distributing software throughout the world.
        jGnash is designed to support many nationalities, so control of distribution would be become very difficult if
        encryption was integrated.

        If you do have the desire to encrypt your jGnash data, the best choice is to use the encryption capabilities
        of your operating system or
        install a freely available third party encryption tool.
    \end{mdframed}

    \subsection{File Types}
    \label{subsec:fileTypes}
    jGnash supports different file types for storing data.
    File types can be easily changed by using the \menu{File > Save As} command and naming the new file with the
    appropriate file extension.

    The current file will be saved in the new format and automatically opened.

    Regardless of file type used, jGnash automatically saves the data if changed every 30 seconds to minimize the
    chance of accidental data loss.

    \notebox{
    If you are using jGnash in the Client/Server mode, all changes are committed immediately.
    }

    \subsection{H2 and HyperSql SQL Databases}\label{subsec:h2-and-hypersql-sql-databases}
    An H2 or HyperSql SQL database is required when using the client / server functionality of jGnash.
    jGnash embeds the database server so that no additional configuration or installation of software is required
    to use a relational database.

    The relational database may be used for a single user.
    If startup and shutdown performance is important to you, then the binary file format described below is a better choice.

    The advantage of the relational database outside the requirement for client / server capability is the ability
    to use several available tools to browse
    and query your jGnash data.
    Also, a relational database is more tolerant of system crashes or power outages vs.\ use of an XML or Binary file.

    The disadvantages of the relational database is a bit slower operation, larger file sizes, and increased memory consumption.

    If using the H2 Database and operating over a network using Client/Server mode, you have the option of enabling
    encryption for network communication.
    This will not encrypt your database file.
    See the Command line options for specifics.

    \notebox{
    The default administrator for a jGnash relational databases is JGNASH and is not configurable at this time.
    }

    \subsection{XML File}\label{subsec:xml-file}
    XML file format is human readable and easily read by other applications at the expense of a considerably larger file size.
    Memory usage is less when using the XML file format, but certain operations may take longer.
    The advantage of the XML file is easier parsing and manipulation of the file using another program external to jGnash.
    If you have a large amount of data, jGnash will use less system memory when using the XML file format.


    \notebox{
    The XML format is also used for creating automatic backups of jGnash files if enabled.
    }

    \subsection{Binary File}\label{subsec:binary-file}
    The binary file format is the most compact file format and will open and close the quickest.
    This is the recommended file format if you do not need client / server functionality and you are using a laptop
    or a workstation with a UPS\@.

    \newpage
    \section{Accounts}\label{sec:accounts}
    Accounts are what you use to organize how you save and spend your money, and where it comes from.
    Account structures can be changed to organize the display of information to suit your specific needs.

    Typically, you will have a separate jGnash account for each savings, checking, investment account, etc.\
    that you have at a financial institution.
    Accounts can be organized under "placeholder" accounts to add different levels of organization.

    For example, if you would like to see a summary of your accounts by financial institution, you can create a placeholder
    account that represents a single financial institution and group your savings and checking accounts from that particular
    institution.

    Maybe you want to see all savings accounts grouped together and checking accounts grouped together.
    It's just a mater of creating placeholder accounts for checking and savings and placing the respective accounts under them.

    The account structure can be easily changed at any time with the exception of removing accounts after they have
    transaction in them.
    Transactions must be manually deleted from an account before you can remove it.

    %\newpage
    \subsection{Creating Accounts}
    \label{subsec:creatingAccounts}
    A new account is created by clicking on the \keys{New} button in the toolbar of the account list view shown below.

    \begin{figure}[h]
        \caption{New Account Button}
        \includegraphics[width=0.4\linewidth]{images/new-account}
    \end{figure}

    A new dialog box will be displayed that allows you to create a new account to suit your particular needs.

    \begin{figure}[H]
        \caption{New Account Dialog}
        \includegraphics[width=0.55\linewidth]{images/account-dialog}
    \end{figure}

    \subsubsection*{Account Information}
    \begin{description}[style=nextline]
        \item[Name]
        The name for the account.
        The name will be used to identify the account when creating transactions.
        Account names do not have to be unique, but may not be left blank.
        \item[Description]
        A description for the account.
        This field may be left blank if desired.
        \item[Account Number]
        The Account Number field is generally used for storing the account number provided by your financial institution.
        This field may be left blank if desired.
        \item[Account Code (General Ledger Code)]
        The Account Code field will default to a value of 0 and must be a numerical value.
        This value will also control the displayed order of accounts within the same branch.
        If the account code is left as a value of zero, sort order will be deferred to an alphanumeric sort against the account names.
        Best practice is to either use an assigned Account Code for all accounts within the same branch or leave the
        code as a value of 0 for all accounts.
        Otherwise, sort order may appear to be random.
        \item[Bank ID]
        This is the identification number of the financial institution and is generally used as an identifier when importing OFX files.
        This field may be left blank if desired.
        \item[Currency]
        The currency for the account.
        Account currencies cannot be changed if the account contains transactions.
        \item[Securities]
        This button will display a dialog that allows you to add and remove the available Securities for the account.
        This applies only to Investment accounts.
        \item[Account Type]
        Account Type determines what the account can be used for and the type of transactions that can be created.
        An Account's type may be changed after it is created to another type only if it is similar in function.
        For example, an Investment Account can be change to a Mutual Fund account and vice-versa, buy they may not be
        changed into Bank or Credit accounts.
        \item[Account Options:Locked]
        If selected, the account will be locked and further changes will be prevented.
        This is useful to lock accounts that have been closed while retaining your historical data.
        \item[Account Options:Placeholder]
        If selected, child accounts may be placed under this account and the account will not accept transactions.
        This is useful for organizing accounts into a hierarchical structure.
        \item[Account Options:Hide Account]
        If selected, the account is hidden from view if the hide account filter is enabled.
        \item[Account Options:Exclude From Budgets]
        If selected, the account is hidden from view of the budgeting tool.
        \item[Parent Account]
        Clicking on this button will display a dialog that lets you select the account this account resides under.
        If you want the account to be placed at the top most level, then choose the Root account as the parent.
        Parent accounts may be changed as needed to suit your needs.
        \item[Notes]
        Room for extra information about the account if desired.
    \end{description}

    At any later time, the \keys{Modify} button may be used to change an existing account.
    If the account contains transactions, certain options may not be changed.

    \notebox{
    jGnash does not have a place to specify an opening balance in keeping with correct practices for double entry accounting.
    \newpage
    \bigskip
    \textbf{However, it is still possible to set and opening balance:}
    \newpage
    \bigskip
    To add an opening balance, create a transaction with an appropriate deposit or withdrawal against an
    Equity Account of your choice.
    Most people will choose or create an "Opening Balance" account.
    \newpage
    \bigskip
    This follows well known accounting practices.
    \newpage
    \bigskip
    \textit{If you are not concerned about correct accounting practices, you can create an Adjustment Transaction instead.}
    }


    \section{Account Types}
    jGnash allows use of several account types to make organization easier.
    The account type chosen can have a significant impact on how reports are generated and displayed, and the types
    of transactions you can create.

    \subsection{Asset}
    Asset accounts are intended to be used to track the value of durable items such as houses, cars, boats, collections, etc.\
    Value of items can be adjusted over time against Income accounts to show gain or loss of value.
    If you were to sell an item and convert it cash, the sale of the item can be tracked against the Asset accounts containing the item.

    \subsection{Bank}
    Bank accounts are used for the savings accounts you would have at a bank.

    \subsection{Cash}
    Cash accounts represent the cash you carry with you.
    Cash accounts are also good for representing deposits and withdrawals from Flexible Spending Accounts.

    \subsection{Checking}
    Checking accounts are used for the checking account you would have at a bank.

    \subsection{Credit}
    Credit accounts are used to record purchases and payments made to a credit card account.
    Credit accounts are primary used for short-term liabilities and great for representing overdraft and line of
    credit accounts at banks.

    \subsection{Equity}
    Equity accounts are used to record opening account balances against.
    Typically, you will have only one Equity account.
    Equity accounts are representative of another accounts net worth at the time you begin tracking it's value.

    \subsection{Expense}
    Expense accounts are used to record expenses such as food, utilities, taxes, investment expenses, etc.

    \subsection{Income}
    Income accounts are used to record income such as salary, dividends, investment income, etc.

    \subsection{Investment}
    \label{sub:investaccount}
    Investment accounts are used to buy and sell Securities.
    Investment accounts can be used to track 401k, IRA's, etc.
    Please see \hyperref[ch:securities]{Securities} for details specific to setting up Securities.       

    \begin{itemize}
        \item Investment Accounts have a cash balance if you buy or sell transactions against the account.
        \item Investment purchases and sales fees can be made against the cash balance of the investment account or
        other specified accounts.
        \item Multiple investment fee entries per transaction may be entered.
        \item Multiple gains/loss entries per transaction may be entered.
        \item Investment Accounts support multiple securities.
        \item Investment Accounts can be used to model an on-line brokerage account.
    \end{itemize}

    \subsection{Liability}
    Liability accounts are used to track long term loans or liabilities.
    Liability accounts have the added feature of allowing you to set-up a loan payment that takes some of the
    effort out of entering periodic loan payment transactions.

    \subsection{Money Market}
    Money Market accounts are typically a high interest yield account with withdrawal rules or limitations.
    They are generally used for long term savings accounts with the intent of keeping your cash readily accessible.
    Please see \hyperref[ch:securities]{Securities} for details specific to setting up Securities.

    \subsection{Mutual Fund}
    Mutual fund accounts are a specialized version of an Investment account and generally used to track mutual fund type
    investments.
    Please see \hyperref[ch:securities]{Securities} for details specific to setting up Securities.

    \subsection{Simple Investment}
    Simple investment accounts are for investments where you do not actively manage or are able to track purchases and
    sales of securities.
    The typical scenario would be a company pension plan that only provides cash balance information.
    Sometimes, these types of investment accounts are called Annuities or Guaranteed Retirement Accounts

    \subsection{Root}
    The root account is the top level account that holds all other accounts.
    You cannot remove or modify the root account.
    Normally, it is not visible unless you are changing the account structure.

    \section{Entering Transactions}
    jGnash follows well know double entry accounting practices while reducing the complexity of transaction entry.

    If you are familiar with other personal finance applications, you may notice that jGnash uses income
    and expense accounts instead of income and expense categories. Functionally, there is no difference,
    other than jGnash allows you see a detailed transaction register of the income and expense accounts as
    easily as you would look at your bank accounts.

    When creating your file, if you selected the available default accounts, you will have income and expense
    accounts to work with.
    These accounts can be changed to suit your needs at any time.

    The benefit of double entry accounting in jGnash is the ability to use reports and charts to see where your
    money is going and coming from.
    Single entry transactions hide those details and make tracking difficult and tedious.

    \begin{mdframed}[style=info]
        For simplicity, the terms used here to discuss changes in account balances will be \textbf{Increase} and \textbf{Decrease}.

        Business accounting practices use the Credit and Debit terms which can appear to be backwards depending on account type
        and cause confusion for those who are not familiar.
    \end{mdframed}

    \subsection{Common Transaction Properties}
    The properties below are common to all basic transaction types.  \textit{Investment transactions will
    require additional information not covered in this section.}

    \subsubsection*{Payee}
    The Payee is almost always the name of a person or business. The Payee field may be empty, but
    it is good practice to always use a payee for a transaction entry. jGnash will automatically learn
    and auto complete a transaction using the Payee field. This helps aid speed of transaction entry and
    helps to establish consistency making searching and filter much easier.

    \subsubsection*{Number}
    A transaction may be assigned a number or abbreviated description.
    If working with a Checking account, the number is almost always the same as the check number you wrote.
    Use of a check number also makes reconciliation against a bank statement easier.

    \subsubsection*{Date}
    Every transaction has a date that is usually the date the transaction occurred.
    Some people will choose to edit the transaction to match the posting date.

    Please review the \hyperref[subsec:dateFields]{Date Fields} shortcuts that may be used to save a significant
    amount of time when selecting and editing dates.

    \notebox{
    jGnash also maintains an internal timestamp for the actual date and time the transaction was created or last modified
    for auditing and tracking purposes.
    }

    \subsubsection*{Memo}
    The Memo is a brief description of the transaction so you can remember what it was.
    jGnash will automatically learn the Memos you commonly use.
    This helps aid speed of transaction entry and helps to establish consistency making searching and filter much easier.

    Split transactions have an option to Concatenate or automatically merge the Memos of the split entries.

    \subsubsection*{Amount}
    This is the transaction amount. If entering a multi-currency transaction, this field will be expanded to handle
    the correct exchange of currencies.

    The \hyperref[subsec:numberFields]{Number Fields} section details how to enter amounts using mathematical operations
    in the same manner as a calculator.

    \subsubsection*{Reconciliation State}
    The Cleared CheckBox at the bottom of the register form is a tri-state box that allows you to toggle through
    all three of the reconciliation states described below.

    \paragraph*{A jGnash transaction has three reconciliation states}
    \begin{description}[style=nextline]
        \item[Not Reconciled]
        The transaction has not been reconciled against a bank or online statement
        \item[Cleared]
        The intended use is for manual reconciliation of a specific transaction prior to a full reconciliation for the period.
        \item[Reconciled]
        The transaction has been reconciled manually or through use of the Reconciliation tool
    \end{description}

    \subsubsection*{Transaction Attachments}
    jGnash allows the attachment of an image file to the transaction using the chain link button at the bottom of the
    transaction form. The attachment will be moved to a managed directory call \directory{attachments} located in the
    same directory as your jGnash file. The eye button allows you to view the attachment and the broken chain loop
    button allows for removal of the attachment.

    \subsection{Double Entry Transactions}
    A double entry transaction follows standard accounting practices. A double entry transaction will always
    increase the balance of one account and decrease the balance of another account. The amount of the change will
    always equal and opposite in value.

    \needspace{5\baselineskip}
    \begin{mdframed}[style=info]
        Double entry transactions using accounts with different currencies will not be equal and opposite numerically due
        to exchange rates between currencies. jGnash automatically handles the difficult part of the currency exchange
        within the transaction.
    \end{mdframed}

    A double entry transaction normally begins within the register of some type of Cash or Bank account.
    In the example below, a Withdrawal is being made from a Cash account. The balance of the account will be decreased
    by \texttt{25.67} and the balance of the \texttt{Expense Accounts:Automobile:Fuel} will be increased by \texttt{25.67}.

    The account to withdraw funds from (Decrease) or deposit to (Increase) is selected in the Account ComboBox.

    \begin{figure}[h]
        \caption{Double Entry Transaction}
        \includegraphics[width=0.9\linewidth]{images/basicDoubleEntry}
    \end{figure}

    \subsection{Split Entry Transaction}
    A Split Entry Transaction is a Double Entry Transaction, but the amount is split across multiple accounts.

    A Split Entry Transaction is started by clicking the \menu{Splits} Button. A dialog will be shown that allows an entry
    for multiple accounts to be made.

    \begin{figure}[h]
        \caption{Split Transaction Entry}
        \includegraphics[width=1.0\linewidth]{images/splitTransactionEntry}
    \end{figure}

    Notice the \menu{Concatenate Memos} Checkbox. If the Checkbox is selected, the Transaction memo will be a generated list
    of the unique memos of the split entries. This saves entry time for the Transaction and reduces the file size.

    Editing of the Transaction amount will be disabled as it is the sum of the Entries. The Memo field will be disabled
    as well if the \menu{Concatenate Memos} in the Split Transaction dialog was selected.

    If a memo is not entered for a Split Transaction and the \menu{Concatenate Memos} button is not selected, the
    Transactions register will display the memo of the first Split Entry.

    A Split Transaction may consist of multiple Deposits and Withdrawals as shown in Figure~\ref{fig:mixed-split-trans}. 
    This is a typical example of a paycheck where insurance and taxes are deducted from the gross salary.

    \tipbox{
        Options specfic to reconsiliation of Split transactions will be covered in a later Chapter.
        \newpage
        \medskip
        You have the option of manual reconsiliation, but there are other ways to speed up the process.
    }

    \begin{figure}[h]        
        \caption{Mixed Split Transaction Entry} \label{fig:mixed-split-trans}
        \includegraphics[width=1.0\linewidth]{images/mixedSplitTransaction}               
    \end{figure}

    \subsection{Single Entry / Adjustment Transaction}\label{subsec:single-entry-/-adjustment-transaction}
    A Single Entry Transaction is primarily used to adjust \textit{(fudge)} an Account balance when you simply do not
    have the information to correctly balance the account.

    In Figure~\ref{fig:adjustment-trans}, you can see a negative amount of \texttt{-1.23} was used to decrease the balance of the account.

    \begin{figure}[H]
        \caption{Adjustment Transaction Entry} \label{fig:adjustment-trans}
        \includegraphics[width=1.0\linewidth]{images/adjustmentTransaction}
    \end{figure}


    The half black and half white circular button at the bottom of the form is a utility to help convert the
    Single Entry transaction into a Double Entry Transaction after you find the needed information.

    \subsection{Transfer Transaction}
    The Transfer tab provides a slightly faster way to move money between accounts without requiring as much information.
    If go back to edit the Transaction, the edits will be performed within the Deposit or Withdrawal forms.
    
    \section{Investment Accounts}
    Investment Accounts need to be populated correctly if the Portfolio report is to work correctly.

    \begin{itemize}
        \item Accounts with prior history should purchase securities from the "Opening Balance" Equity account. Ensure the quantity of
        securities is correct and use the current price.
        \item Cash balances should be transfers from the "Opening Balance" Equity account.
        \item Dividends, sells, etc.\ should occur after opening securities and cash balances are established.
    \end{itemize}

    Please see the \hyperref[ch:securities]{Securities} chapter for details specific to setting up Securities.

    \chapter{Reminders}
    If you find yourself recreating the same transaction on a regular basis, Reminders can be created to automate the process.

    \includegraphics[width=0.8\linewidth]{images/reminders}

    The table shown above displays all the Reminders you have created with some basic information.

    \begin{itemize}
        \item The \textbf{Last Posted} column shows the last date the Reminder was processed. It will be empty for a new Reminder.
        \item The \textbf{Due} column shows the next due date if a configured end date has not been reached.
        \item The \menu{Modify} button is for editing the selected Reminder.
        \item The \menu{Execute Now} button will process the selected Reminder if it is due.
        \item The \menu{Check Reminders} button looks for any pending Reminders and displays them. If there are not
        any pending Reminders, the dialog will not be displayed.
    \end{itemize}

    \section{Notification of Reminders}
    The pending Reminders dialog will be shown shortly after jGnash has been started or when manually checked
    as described earlier.

    \includegraphics[width=0.8\linewidth]{images/remindersPopupDialog}

    \begin{itemize}
        \item The \menu{Remind Me Later} button lets you to snooze the dialog for a selectable period of time if you do not want to
        process the transactions at that time.
        \item The \menu{Acknowledge Selected} button processes \textit{only} the transactions marked in the \textbf{Approve} column.
    \end{itemize}

    Below the \textbf{Approve} column are buttons to make it easier to mark a large number of transactions for approval.

    \section{Creating New Reminders}

    Creating new Reminders is a process of filling in some basic information, creating your transaction, and configuring the
    frequency of the Reminder.

    An Account must be selected for the Reminder and fields exist to provide a description and additional notes.
    A transaction is optional
    A Reminder without a transaction will simply be a Reminder to do something.

    The typical approach for assigning the Account is to use a bank or cash account of some sort. When you create the
    transaction, it will be against an expense or income account unless it is a periodic transfer between accounts.

    However, if you prefer to organize Reminders by expense type or account, you can pick an expense or income
    account and create the transaction against a bank account.
    Regardless of approach, be consistent to avoid confusion later.

    \includegraphics[width=0.8\linewidth]{images/remindersNewDialog}

    Configuring the Frequency of a Reminder is easy despite there being a lot of options. The \textbf{Date of first payment}
    field establishes the first date the Reminder process starts as well as the day of the month it occurs on.

    The \textbf{Enabled} check box will be selected by default. Unselecting it will pause the Reminder should you want
    to disable it for a period of time instead of deleting it.

    The \textbf{Month} frequency tab provides an option of \textbf{Month By Day} or \textbf{Month by Date}.
    As an example, selecting \textbf{Month By Day} indicates you may want a Reminder to occur every 2nd Tuesday of the month.
    Selecting \textbf{Month by Date} would indicate you want a Reminder to occur of the 15th of every month.

    Creating a transaction for a Reminder is identical to creating a transaction in the register. The Date field will be
    ignored and replaced by the recurrence date of the transaction defined in the Frequency section.

    \includegraphics[width=0.85\linewidth]{images/remindersNewTransaction}

    The \textbf{Entry} section of the New Reminder form gives you option to automatically enter a new transaction a defined number
    of days prior to the date defined by the Frequency section. When enabled, you will not see the Notification dialog.
    Typical use is for automatic withdrawals or deposits into an account.

    \section{Tips}

    \subsection{Quick Generation of new Reminders}
    New reminders may be created from an existing transaction.

    Within the Register, right click on a transaction to display the context menu.
    From the context menu, the \menu{Create new reminder} command will generate a new reminder using the existing
    transaction as a template and display the New Reminder Dialog.
    You will need to ensure the Frequency parameters are correct and simply click the\menu{OK} to save it.

    \includegraphics[width=0.8\linewidth]{images/remindersQuickTransaction}

    \subsection{Special Dates}
    Be careful when creating Reminders near the end of the month with a Month frequency.

    \begin{itemize}
        \item What happens if you select February 29th (Leap Day)?
        \item Do you receive a paycheck every two weeks (26 pay periods) or twice a month (24 pay periods)?
    \end{itemize}

    jGnash makes the best attempt to ensure correctness of dates, but minor date shifts can occur should you select leap days,
    the 1st day of a 5 week month, etc.

    \chapter{Budgets}

    jGnash has a budgeting feature that makes it easy for you to define spending and income goals by account and bump those goals up against your actual transactions.
    A compact graphical overview of each budgeting period is provided to highlight how well you are following your budget based on selectable periods.

    Tracking how well you follow your budget can be an eye opening experience and can lead to better financial health.

    \notebox{
    jGnash budget periods follow rules established by ISO 8601. Weeks begin on Monday and the first week of the year may begin
    with the last few days of the prior year.
    }

    \section*{Budget Features}

    \begin{itemize}
        \item Multiple budgets are supported and may be copied making it easy to try out different scenarios and create year specific budgets if desired.
        \item Allowed accounts for budget are limited to Income and Expense accounts.
        \item Accounts may be excluded from budgets by setting the exclude flag in the account properties dialog. Sub-accounts will not be displayed if the parent account is excluded.
        \item The reporting period for budgets may be daily, weekly, bi-weekly, monthly, or quarterly and can be changed as needed.
        \item The per account budget goals may also be entered in daily, weekly, bi-weekly, monthly, or quarterly periods and are independent of the budget reporting period.
        \item The budget may be exported to a spreadsheet.
    \end{itemize}

    \tipbox{
    The reported period of the budget is independent of the per account budget goal period.
    \newpage
    \medskip
    Example: Your salary is paid in bi-weekly intervals, but you want to see your budget reported by month.
    You can change the period for the income account to weekly or daily and enter your salary.
    \newpage
    \medskip
    When reported by month, jGnash automatically handles the difference in the periods and distributes your bi-weekly
    salary income across monthly boundaries.
    }

    \section{Graphical Overview}

    The main budget panel is shown below.

    \includegraphics[width=1.0\linewidth]{images/budget-overview}

    The width of the account column is adjustable by placing the cursor between the Account header and the period header
    columns and then clicking and dragging the mouse cursor right or left.

    At the top of the panel, a toolbar exists that allows you to change how much information is displayed, modify the
    budgets, and export the active budget to a spreadsheet.

    The budget drop down list lets you quickly select between different budgets you have created.
    The \menu{Budget Manager} button displays a dialog that let you create, duplicate, and delete budgets.

    The year spinner allows you to bump the selected budget up against the selected year's transaction data.
    The selected year also effects the calendar periods when editing period amounts.

    \subsection{Properties}\label{subsec:properties}
    The \menu{Properties} button will display the dialog shown below with various options for the active budget.

    The period used for the budget display can be changed in this dialog as well as the budget description.
    You may also select the account groups that are visible for the selected budget.

    The \textbf{Start Month} property sets the month you begin a new budget. For most, it's the beginning of the year, but
    others may prefer a month such as April which is tax season for some. If you are new to jGnash and you've started
    mid-year, then the month you being is a good start. The \textbf{Start Month} may be changed at any time without negatively
    impacting your budget goals.

    Regardless of the \textbf{Start Month} selected, jGnash will use a rolling 12 month period based upon the \textbf{Start Month} you
    have selected.
       
    \begin{figure}[h]
        \caption{Budget Properties}
        \includegraphics[width=0.45\linewidth]{images/budget-properties}
    \end{figure}

    The rounding method and number of decimals used to report budget results may be changed.
    The default Rounding Mode is \textit{Floor} which will round up expenses and round down income values which is a conservative
    view of current income and expenses.

    \begin{table}[H]
        \begin{tabular}{|l|l|}
            \hline
            \textbf{Mode} & \textbf{Description} \\
            \hline
            \hline
            Ceiling & Rounds towards positive infinity \textit{(Optimistic View)} \\
            \hline
            Down & Rounds towards zero \\
            \hline
            Half Down & Rounds towards nearest neighbor or down if equal distance \\
            \hline
            Half Even & Rounds towards nearest neighbor or towards even if equal distance \\
            \hline
            Half Up & Rounds towards nearest neighbor or up if equal distance \textit{(Banking)} \\
            \hline
            Floor & Rounds towards negative infinity \textit{(Conservative View)} \\
            \hline
            Up & Rounds away from zero \\
            \hline
        \end{tabular}
        \caption{Rounding Modes}
    \end{table}

    Reported values are rounded based on the selected value for \textbf{Scale}. Negative values for \textbf{Scale} will round to the
    left of the decimal point while positive values round to the right of the decimal point.

    The upper limit of the \textbf{Scale} option is limited by the largest \textbf{Scale} value of the Currencies you have configured.
    \newpage
    \subsection{Budget Management}
    Double clicking on an account name to the left of the panel will display a dialog that allows you to change the account
    specific budget period and period amounts.

    \includegraphics[width=0.8\linewidth]{images/budget-goal-dialog}

    The Smart Fill panel may used to enter repeating patterns or fill in the amounts automatically based on the last 12 months.
    Alternatively, you may directly edit the amount of each period by clicking and typing in a table cell.

    The per account budget amounts as well as the \textbf{Change} and \textbf{Remaining} values are hierarchical in that the values of the child account
    are summed and are added to the parent account. If a parent account is not configured has a placeholder, it may also be assigned
    period goals that are inclusive of any children.

    At the bottom of each reported budget period, a summary by account group is displayed.
    To the right, a summary by account is displayed.
    The summary's made be disabled if desired by unselecting the appropriate check boxes.

    The \menu{Export Spreadsheet} button will export a file to your choice of an \texttt{.xls} or \texttt{.xlsx} file.
    The exported spreadsheet does contain formulas which makes it easier to manipulate the file externally.

    \subsection{Budgeting Tips}

    When planing a budget, you need to consider how you spend and receive your money versus how you want to report your budget.

    jGnash has to make assumptions when entering per account period amounts.
    Internally, jGnash is keeping a list of 366 days (365 + 1 leap day) per account with the list starting at the first
    calendar day of the year.

    When a period goal for an account is entered, the amount is averaged across each day of the period.
    Entry of amounts is also sensitive to the current year.
    If you select Monthly for the account period, the monthly boundary for days is established by the current year calendar
    months and the amount is then averaged across the number of days per each month.

    Averaging of periods has an impact on how exact the tracking of your budget is.
    If you choose to enter a monthly average for income, but are paid on certain days on the month, your budget will show
    slight variations through the year.

    If you want the budgeted vs.\ Remaining amounts to be exact for a particular account, then you will want to set the
    account period to be Daily and take the effort to enter your daily amount goals.

    Minor discrepancies can occur during leap years due to the extra day. Budgets are typically dynamic due to continuous
    changes in income and expenses which means you will naturally address these small discrepancies as you maintain your budget

    You will not be able to export a spreadsheet when the report period is daily due to memory requirements and limitations
    of some spreadsheet applications.

    \chapter{Reconciliation}
    Reconciliation is a simple and visual process of matching up the transactions listed in an account's register against a
    paper or electronic statement provided by a financial institution.
    If differences do exist, then any missing or erroneous transactions must be addressed until the differences are resolved.
    Statements should come from you bank, investment broker, credit card issuer, etc.\ on a periodic basis.

    Periodically reconciling an account helps ensure transaction entry errors do not creep in over time.

    \tipbox{
    Reconciliation is also a great tool for monitoring your accounts for fraudulent transactions.
    }

    Accounts may be reconciled using a manual process or using the Reconciliation Wizard.
    For any given reconciliation period, using a combination of both methods will be difficult.

    Regardless of the Reconciliation process used, it is good practice to reconcile all accounts periodically.

    \section{Basics}
    Every transaction entry will have two independent reconciliation states that applies to both of the related
    crediting and debiting accounts.
    Split transactions may have even more reconciliation states depending on how many accounts it touches.
    Reconciliation states are explained a bit later in this chapter.

    The default assumed reconciliation states can be configured depending on your preferred method of reconciliation.
    Taking time to understand these options is important for a successful reconciliation process.

    \tipbox{
    If this is your first time reconciling an account and you have prior transaction history with a mix of
    Reconciled and Cleared transactions, you may need to manually reconcile prior transaction history.
    }

    \subsection{Reconcile Settings}

    jGnash makes it easier to manage reconciliation by providing some options described below.
    These can be access in the \textbf{Register} options tabs using the \menu{Tools > Options} menu.

    \includegraphics[width=0.8\linewidth]{images/reconcile-options}

    Unless you have very specific needs, it is recommend that you choose to have the option \textbf{All transaction accounts have
    same reconciled state} or \textbf{Automatically reconcile Income and Expense Accounts} selected as the default.

    Selecting \textbf{Automatically reconcile Income and Expense Accounts} requires a bit more work on your part in that you are
    required to reconcile all institution statements, but it will not create any issues when transferring between accounts.
    \textit{This is the recommended option if you are using the Reconciliation Wizard.}

    Selecting \textbf{All transaction accounts have same reconciled state} will reduce the effort of reconciling transactions,
    but can create problems reconciling transfers between bank accounts later. If you manually reconcile your accounts and
    do not use the Reconciliation Wizard, this option saves a significant amount of work at the risk of making an assuming
    a transaction occurred correctly between two institutions.
    \textit{Use of this option in conjunction with use of the Reconciliation Wizard can create problems with bank transfers when
    you reconcile both accounts.}

    Choosing to disable automatic reconciliation will require you to reconcile Income and Expense accounts for which you
    may not have been provided a reconciliation statement.

    \subsection{Reconcile States}

    jGnash transactions have three reconciliation states that are presented in order below:

    \begin{description}[style=nextline]
        \item[Not Reconciled]
        The transaction has not been cleared or reconciled.
        \item[Cleared]
        The transaction has been marked by the user to have been cleared during a manual or unfinished reconciliation process.
        A transaction may be marked as cleared to draw attention to it without impairing use of the reconciliation wizard.
        \item[Reconciled]
        The transaction has been automatically or manually reconciled.
    \end{description}

    \warningbox{
    Manually marking transactions is not recommended if you are going to use the Reconcile Wizard.
    }

    \section{Manual Reconciliation}
    Manual reconciliation is the process of individually comparing the account register against the institution provided
    statement and marking the matching transactions as reconciled.

    The downside to manual reconciliation is not all checks and balances are performed against the reported opening and
    closing balance for a given period. This increases the likely-hood of missing a recorded transaction or incorrectly
    entered amount.

    To manually mark a transaction as reconciled, use the context menu in the register to display options to change the
    reconciled state.

    \includegraphics[width=1.0\linewidth]{images/manual-reconcile-context}

    \begin{mdframed}[style=info]
        Use of the context menu is currently the only means of marking a transaction as reconciled other than using the
        reconciliation tool. It may also be used to clear transaction erroneously marked as reconciled.
    \end{mdframed}

    Transactions may also be marked as \textbf{Cleared }through the transaction form.
    Some users may prefer to clear certain transactions manually during a given period to draw attention to them.
    \textbf{Cleared} transactions will still be visible within the Reconciliation Wizard if used later while manually
    \textbf{Reconciled} transactions will not.

    \includegraphics[width=1.0\linewidth]{images/transaction-form}

    \section{Reconciliation Wizard}
    Use of the Reconciliation Wizard helps to simplify the reconciliation process by comparing opening and closing balances
    reported by the institution against the sum of the transactions as you mark them as reconciled. You receive
    instantaneous visual feedback as you mark transactions, and at the end of the process you should have a net difference
    of zero.

    \tipbox{
    The Reconcile Wizard has a nice feature that is not immediately obvious. While the wizard is displayed, you can still
    go back to the account register and enter missing transactions, correct erroneous amounts, or modify and delete
    transactions if entered into the wrong account. The Wizard's credit and debit lists are fully dynamic.
    You are not required to exit the Wizard without completing the process if you discover missing transactions or errors.
    }

    he image of the account register shown next is representative of a small but typical reconciliation period.
    The amounts and balances shown correspond with the other images as the Reconcile Wizard is explained in this chapter.
    Refer back to this image as necessary for clarification.

    Take note of the \texttt{Reconciled Balance:\$2,614.43} and that it is the last transaction marked as reconciled.
    Also, take note of the transaction dated \texttt{03/25/14} and the corresponding balance of \texttt{\$1,369.70}.
    You will see these same values later in the Reconcile Settings dialog.

    \includegraphics[width=1.0\linewidth]{images/reconcile-register}

    The Reconcile Wizard is started by using the context menu in the Account List, or by clicking the \menu{Reconcile}
    button in the transaction register.

    A small dialog will be shown requesting some information.

    \begin{description}[style=nextline]
        \item[Statement Date]
        This is the closing date for the reconciliation period.
        This should be reported on your account statement.
        The date will typically be the end of the month, but may be different due to institution or locale rules.
        Transactions entered after this date will not appear within the Reconciliation Wizard
        \item[Opening Balance]
        The opening balance should also be provided by your institution and should be equal to the closing balance
        of the last reconciliation period.
        In your account register, this will also be the account balance of the last reconciled transaction.
        \item[Ending Balance]
        This amount should also be provided by you institution.
    \end{description}

    \textit{These values should be provided to you by your banking institution in paper or electronic format and
    it's important these values are entered correctly, otherwise balances will not zero out.}

    Pay special attention to the account type, the selected option for \textit{Reverse Displayed Account Balances}
    and if you are entering a positive or negative opening and ending balance.

    \includegraphics[width=0.4\linewidth]{images/reconcile-settings}

    After clicking the \menu{OK} button, the settings dialog will be replaced by a
    dialog showing all of the transactions prior and inclusive of the statement date
    that have not been marked as Reconciled. \textit{Transactions marked as cleared will be shown.}

    The next step is to go through the institution provided statement and mark every
    matching transaction as reconcilable by clicking on the transactions. As you
    click each transaction, totals will update and you should see the
    \textbf{Difference} value approach zero. The symbol in the \textit{Clr} column will
    also change when marked as \textbf{Reconciled}. When the \textbf{Difference} is zero,
    the \menu{Finish} button will become active. Transactions marked as cleared will
    also need to be selected if they are to be reconciled.

    It is not unusual to find transactions that go unmarked for reconciliation near
    the end of the statement period. These are transactions you have entered that
    were not processed through the system fast enough to show up on your statement
    and impact your account balance. Simply ignore these transactions and they will
    be captured at the start of the next statement and reconciliation cycle.

    \newpage
    \includegraphics[width=0.8\linewidth]{images/reconcile-dialog}

    Clicking on the \menu{Finish} button will close the dialog and will mark the selected transactions as Reconciled.
    Depending on the number of transactions and type of file format being used, it could take awhile for the changes
    to be saved.
    A wait message will be displayed during the change process.

    What do you do if you have marked all transactions as reconciled and the difference is not zero?

    \begin{itemize}
        \item Not all paper and electronic statements clearly identify fees, earned
        interest, etc.\ Make sure you have captured these transactions.
        \item Were any transactions amounts entered incorrectly?
        \item Transactions manually marked as Reconciled during the statement period will
        not show in the transaction columns and are guaranteed to throw off balances.
        Mark the transactions as \textbf{Cleared} instead.
        \item Do you have your Reconcile Settings configured appropriately for the process
        you are using? If in doubt, use \textbf{Automatically reconcile Income and Expense Accounts}.
        \item Incorrectly entered opening and ending balances will cause errors in calculated balances.
    \end{itemize}

    If you need to to exit the Reconciliation Wizard before finishing, the \menu{Finish Later} button may be used.
    This will close the dialog and mark selected transactions as \textbf{Cleared}.
    This makes it easy to restart the process with transaction you have already marked as reconcilable identified.
    You will still need to re-select those transactions when you restart the process.

    \chapter{Securities}
    \label{ch:securities}
    Securities must be managed when entering Investment Transactions within jGnash.  
                    
    At a minimum, a Security is assigned a unique Symbol, a Reported Currency and a Scale \textit{(the number of decimal places 
    for entering and reporting prices)}.  

    Tracking of prices and historical information for the Security is optional and not required to enter
    investment transactions.  However, the reported \textbf{Market Balance} of an account will more accurate if prices are entered.
    
    The same Security may be used for multiple accounts.
    
    \tipbox{
        The term Securities is used within jGnash generically for Stocks, Mutual Funds, Bonds and when trading commidities
        such as gold and oil.
    }
    
    
    \section{Create and Modify Securities}
    
    Securities may be created or modified using \menu{Tools > Securities > Create / Modify}.  All of the properties are
    accessible when modifying an existing Security.
    
    To create a new Security, simply fill out the form with the required properties and click the \keys{Apply} button to save.
    
    You can modify an existing Security by selecting it in the list to the left and click 
    the \keys{Apply} button to save the changes.
    
    \begin{figure}[h]
        \caption{Create / Modify Securities}
        \includegraphics[width=0.6\linewidth]{images/createModifySecurities}
    \end{figure}

    \begin{description}[style=nextline]
        \item[Symbol \textit{(required)}]
        The Symbol is typically the Ticker symbol assigned to the Security.  Depending on the Quote Source used,
        the Symbol my be appended with a market or region code. This value may be left empty.
        \item[CUSIP / ISIN]
        A internationally unique identifier for a Security.  Some Quote Sources may require use of this field
        instead of using the Symbol.  At this time, it is not required. 
        \item[Quote Source \textit{(required)}]
        The Quote Source is the online data source chosen to download prices and events.  This is discussed later 
        in the chapter.
        \item[Description]
        A general description for the Security.  This value may be left empty.
        \item[Scale \textit{(required)}]
        This is the number of decimal places used for entry of security prices.  Mutual funds will typically
        be traded at fractional prices.  This allows you to override the number of decimal places used by the
        Reported Currency.
        \item[Reported Currency \textit{(required)}]
        Sometimes called the Base Currency, this is the currency the Security is being traded with.        
    \end{description}        
    
    \subsection{Quote Sources}
       
    jGnash currently supports use of Yahoo and IEX Cloud as online data providers.  
    
    \subsubsection{None}
    The default quote source is \textbf{None}.  If used, all pricing, stock splits and dividends will
    need to be entered manually.
    
    \subsubsection{IEX Cloud}
    Using IEX Cloud as a Data Provider requires opening an account which may be free for use or a paid
    subscription depending on the amount of data being consumed. 
    
    An account can be created at \href{https://iexcloud.io/}{IEX Cloud} \texttt{[https://iexcloud.io/]}.
    After creating your account, the \textit{Secret Key} provided by IEX Cloud must be entered into the 
    Data Providers Options Tab as shown in Figure~\ref{fig:dataproviders}.
    
    The \textit{Secret Key} is stored within your jGnash file.  If the key is not entered,
    jGnash will report an error when attempting to download information.
    
    \begin{figure}[h]
        \caption{Data Providers}
        \label{fig:dataproviders}
        \includegraphics[width=0.6\linewidth]{images/dataProvidersTab}
    \end{figure}
    
    \subsubsection{Yahoo}
    Prices and historical events are downloaded from \href{https://finance.yahoo.com/}{Yahoo Finance} \texttt{[https://finance.yahoo.com/]}.
    Usage is subject to the conditions and terms of the Yahoo website.
    
    Yahoo does not make all Securities available, and it is not uncommon to experience periods of inaccessibility.
    
    \subsection{Security Price History}
    
    Securities history is managed using \menu{Tools > Securities > History}.
    
    The \textbf{Security} of interest is selected at the top of the form.
    A chart is displayed using on the historical closing price.  
    As shown in Figure~\ref{fig:securitieshistorydialog}, periods divided by a stock split
    are shown using a different color.
    
     \begin{figure}[h]                          
        \caption{Modify Securities History}                  
        \label{fig:securitieshistorydialog}   
        \includegraphics[width=1.0\linewidth]{images/modifySecurityHistory}
    \end{figure}
    
    \subsubsection{Price History Panel}
        
    The closing price as well as the daily high, low and trade volume can be filled in and added by clicking
    on the \keys{Add} button.  The \keys{Clear} button will reset the entry form.  Selecting a date in the list
    and clicking the \keys{Delete} button will permanently remove the entry.
    
    At the bottom of the \textbf{Price History Panel} are some convenience buttons to help pare down the amount of data
    that can accumulate over time.  
    
    Thinning out the data can significantly reduce the size of your file and improve application performance.
    
    Please see \hyperref[sec:pricelookup]{Price Lookup and Reporting} Section for specifics of how jGnash uses
    the Price History for reporting and estimating Investment Account balances.
    
    \subsubsection{Event History Panel}
    
    Similar to the \textbf{Price History Panel}, Dividends and Splits may be entered.
    
    Dividends can be added for informational purposes but are not required.
    
    Split entries help to correctly manage the visual trend of the chart.  They are not used or required
    to correctly calculate the value of your Investment Account.
    
    \warningbox{
       When Splits or Diviends do occur, it's important
       to remember to create a Split Transaction within the Investment Account register.  
       Splits and Dividend chart events do not automatically register has transaction events.
    }
    
    \tipbox{
        When entering splits, the value is typicaly greater than 1.  For example, a 2:1 split is recorded as 2.0.
        However, if a 1:2 split or similar occurs (known as a reverse split), a value of 0.5 would be entered.
    }
        
           
    \subsection{Security Price Historical Download}
    
    Historical price information for Securities may be downloaded using \menu{Tools > Securities > Historical Import}.
    
    Shown below Figure~\ref{fig:historicalImport}, you need to specify the start and end dates as well as check the boxes of the Securities you
    want to download the historical information for.
    
    Click the \keys{Start} button at the bottom of the form.  The download and import process can take awhile 
    depending on the file type you are using.  
    
    If desired, you can click the \keys{Stop} button to interrupt the process.  Interrupting the process will not
    undo what has already been downloaded and imported prior to stopping.
    
    \begin{figure}[h]
        \caption{Historical Import}
        \label{fig:historicalImport}
        \includegraphics[width=0.6\linewidth]{images/securityHistoryImport}
    \end{figure}
       
    \newpage   
    \subsection{Price Lookup and Reporting}
    \label{sec:pricelookup}
           
    When reporting balances for \hyperref[sub:investaccount]{Investment Accounts}, jGnash will perform a search
    for the most recent or closet price.
        
    \begin{itemize}
        \item If an Investment Transaction occurs on the same day as a price record, the Transaction price will be used.
        \item For reporting periods such as Months or Quarters, the closest or exact match to the end of the period is used.
        \item  In all cases, the latest price if it's a Transaction or a Historical record is used for the last period 
        of a Report and as well as the reported \textbf{Market Value} of the account shown in the Register.      
    \end{itemize}
      
    Deleting price records near the end of a reporting period can skew the reported interim values.

    \chapter{Importing Transactions}
    Importing of OFX, QFX, MT940, and QIF transaction files is supported within jGnash.
    The files are downloaded manually from your financial institution and then imported.
    jGnash does not currently support an automatic download process.

    QIF is an old format that was not well defined and is generated inconsistently between financial institutions.
    There are two significantly different formats with one consisting of transnational data only and the other format
    attempts to serve as an export / import format with account information for financial applications.
    Use of date formats within the QIF format will vary and can lead to incorrect or failed import of transactions.
    If at all possible, avoid use of QIF files and use an OFX or QFX file.

    OFX and QFX files are a well defined standard and jGnash will easily process these.
    This is the preferred file format for importing transactions into jGnash.
    OFX and QFX files are XML based and human readable in that you can open the file
    with a text editor and easily understand the information contained within.

    MT940 is another well defined standard commonly used between financial institutions.
    The format consists of fields separated by colons and commas with well defined numeric codes.
    MT940 is very compact and efficient to transfer electronically.

    \notebox{
    The Import Process below describes the basic import of transactions.
    jGnash can also process OFX and QFX files with Investment transactions.
    Additional columns will be present and enabled when applicable for these advanced types of transactions.
    }

    \section{Import Process}
    Importing transactions is an easy process that is guided with a simple wizard.
    It begins with \menu{File > Import > OFX / QFX}, \menu{File > Import > QIF}, or
    \menu{File > Import > MT940} and selecting a file that has been downloaded.

    The first step of the import process requires selecting the correct account the
    import file is for. This is the base account all imported transactions will be
    associated with.

    \tipbox{
    When importing OFX or QFX files, jGnash will preselect the correct account after
    it has learned from the first import.
    }

    The second step of the process is reviewing the transactions as shown below.

    \includegraphics[width=0.8\linewidth]{images/importWizard2}

    The first column consists of import status symbols:

    \begin{itemize}
        \item A plus symbol indicates the transaction has been detected as new. Double
        clicking on it will change it to a negative symbol to indicate you do not want
        it to be imported.
        \item An equals symbol indicates the transaction has been detected as a duplicate of
        a manually entered transaction or already imported transaction. Double
        clicking on it will change it to a plus symbol to indicate you want to force
        the import of the transaction.
    \end{itemize}

    The second, third, and forth columns are not editable. The Payee and Memo
    columns may be pre-processed using JavaScript Filters as described later.

    The Account column can be double clicked and changed to suit. The selected
    account will normally be an income or expense account unless it was a transfer
    to another bank account or financial institution. This column is also the
    equivalent to \textit{Categories} for some other financial applications.

    The Amount column can be double clicked and edited. This is normally not
    needed and not recommended, but allowed for some specialized uses when working
    with multiple currencies.

    The \menu{Delete} button may also be used to remove a transaction being imported.

    \tipbox{
    When importing transactions, jGnash uses what is called a Naive Bayes Classifier
    to help preselect the best account for the transaction. Consistent use of
    memo and payee fields will help accuracy of account selection. Accuracy also
    improves with increased quantity of transactions over time.
    }

    The last step of the Import Wizard allows you to review basic information about
    the import before committing to the changes. After clicking on \menu{Finish},
    the dialog will close and jGnash will begin the import process.

    \section{OFX imports}
    Taking the time to correctly configure your jGnash Account properties can help
    reduce the effort needed to automatically identify Accounts when importing OFX
    transactions.

    OFX files may contain enough information to identify transfers between accounts.
    Most financial institutions will correctly identify transactions between checking
    and savings accounts with the bank, but others may include enough information
    when transferring between finance institutions.

    Below is a fragment of the OFX file content defining the transfer. jGnash will
    attempt to match the \texttt{<ACCTID>} identifier on line 4 to an existing Account Number
    that is specified within the properties of your jGnash account.
    \\ % new line
    \begin{lstlisting}[caption={OFX File Fragment}]
        <BANKACCTTO>
        <BANKID>100000009
        <BRANCHID>TEST
        <ACCTID>555555-C01
        <ACCTTYPE>CHECKING
        </BANKACCTTO>
    \end{lstlisting}

    The \textbf{Account Number} is easily changed by modifying the Account properties as described
    in the \hyperref[subsec:creatingAccounts]{Creating Accounts} section of the manual.

    \section{JavaScript Filters}
    When importing transactions from OFX/QFX, mt940, and QIF bank statements
    downloads, transactions may be preprocessed using custom JavaScript files. As
    of jGnash release 2.32.0, the scripts may be used to alter the memo or payee of
    the transactions being imported.

    You have three options for making a JavaScript file available for use:
    \begin{itemize}
        \item Some scripts are included by default. These show up as \texttt{/jgnash/imports/xxx.js}
        in the \textbf{Script} column of the table.
        \item Place the JavaScript file into a \directory{importScripts} sub directory where
        your jGnash data file is installed.
        \item Place the \texttt{.js} file into your home directly where jGnash will find it.
        \begin{itemize}
            \item \directory{\$HOME/.jgnash/importScripts} for UNIX and BSD based operating systems
            \item \directory{USER\_HOME\textbackslash AppData\textbackslash Local\textbackslash jgnash\textbackslash importScripts} for Windows operating systems.
        \end{itemize}
    \end{itemize}

    These import scripts must be specifically enabled for use and the order in which they will be executed can be changed
    using the \menu{Tools > Configure Transaction Imports Filters…} menu as shown below.

    \includegraphics[width=0.9\linewidth]{images/importFilters}

    \newpage
    Below is an example JavaScript file. The first four functions are required for correct operation.

    The first function accepts an \textbf{ImportTransaction} for advanced manipulation of the imported data.

    The second and third functions accept a \textbf{string} and are expected to return a \textbf{string}.

    The \textbf{getDescription} function should return a meaningful description of the script.
    \\ % new line
    \begin{lstlisting}[caption={Example Import Script},language=JavaScript,numbers=left]
        /* Normalizes the case of the payee and memo fields */

        // place holder for the passed ImportTransaction
        var importTransaction;

        // This will be called first and passed the ImportTransaction
        function acceptTransaction(transaction) {

            // does nothing with it in this script
            importTransaction = transaction;
        }

        // This is a required function
        function processMemo(memo) {
            return capitalizeFirstLetter(memo.toLocaleLowerCase());
        }

        // This is a required function
        function processPayee(payee) {
            return titleCase(payee.toLocaleLowerCase());
        }

        // This is a required function
        var getDescription = function (locale) {
            var Locale = Packages.java.util.Locale;
    
            switch (locale) {
                case Locale.ENGLISH:
                    return "Tidy Memo and Payee fields";
                default:
                    return "Tidy Memo and Payee fields";
            }
        };

        // Capitalizes the first letter or a String
        function capitalizeFirstLetter(str) {
            return str.charAt(0).toUpperCase() + str.slice(1);
        }

        // Converts a string to Title Case
        function titleCase(str) {
            return str.replace(/(^|\s)[a-z]/g,function(f){return f.toUpperCase();});
        }
    \end{lstlisting}

    The full jGnash API may be accessed within these scripts for advanced processing capabilities.
    See the \hyperref[sec:javascript]{JavaScript} section of the manual for more details.

    \notebox{
    JavaScript files are expected to be encoded as UTF-8 Files.
    An incorrect file encoding may cause script failures.
    }

    \chapter{Reports}
    A variety of reports exist that present your financial history and status in different ways.
    There are currently three classes of reports available.
    Text reports can be exported and easily imported into a spreadsheet for advanced manipulation.
    Chart based reports may be altered and exported to a graphic file or printed using the context sensitive pop-up menu.
    Tabular type reports may be printed or saved as \textit{PDF}, \textit{XLS}, or \textit{XLSX} files.

    \section{Tabular Reports}
    Tabular reports are displayed in a viewer that allows you to change the page and print or export the report.
    The font size of the displayed report can be changed from the toolbar of the report window.

    The fonts used to display the report may be changed in the \menu{Tools > Options}
    dialog shown below. The \textbf{Proportional} font is typically used for report headers and footers.
    The \textbf{ Monospace} font, also called a fixed-width font, is used to display the reported values.

    \includegraphics[width=0.6\linewidth]{images/font-options.png}

    If a proportional spaced font is chosen for the \textbf{ Monospace} font, numeric report values may not line up
    correctly in the report.

    \tipbox{
    Information on font types as well as a wide selection of freely available fonts can be found on the Internet.
    Once a new font is properly installed in your operating system, it will be available for use in jGnash the next
    time it is started.
    }

    \section{Tips}
    Depending on your operating system, or locale you may need to change the font type and font size to achieve
    the best looking report.
    The font size can be changed on the report toolbar and is remember for each report type.

    \subsection{Nothing displays in the report and I'm not getting any errors}
    Try increasing your font size, and if that does not work, choose a different font.
    Depending on your operation system, fonts may not render correctly at reduced sizes.

    \subsection{I get an error that tells me to reduce my font size}
    The selected font size is too large to display the report correctly.
    You will need to choose a smaller font size.
    Many times, the column heading text may dictate the displayed width of a column.
    Try choosing a proportional font with condensed spacing.
    You may also want to check the default paper size and adjust if needed.

    \subsection{My PDF exports are missing information or don't look correct on different computers}
    Not all fonts are able to be embedded within a \texttt{PDF} file.
    You may need to experiment with different fonts to achieve good portability.
    In most cases, the defaults jGnash chooses will give you good results.

    \subsection{The IRR is not being displayed in my Portfolio report}
    Your investment account may not be setup properly. Ensure that stocks have been added
    or purchased prior to sells, buys, dividends, etc.

    \chapter{Administration}
    Several administration options and tools are provided to help with management of your data.

    \section{File | Save As}
    An open file may be saved as a new file of the same type, or a new file with a new file type.
    To save the file as a different type, you must change the file extension to a supported type.
    Correct file extensions are shown below.

    \begin{table}[H]
        \begin{tabular}{|l|l|}
            \hline
            XML File & .xml \\
            \hline
            Binary File & .bxds \\
            \hline
            H2 Relational Database (B-Tree) & .h2.db \\
            \hline
            H2 Relational Database (MVStore) & .mv.db \textit{(newest H2 format and is more compact)} \\
            \hline
            HyperSql (hslqb) Relational Database & .script \textit{(.lobs, .log, .properties are used as support files)} \\
            \hline
        \end{tabular}
        \caption{File Types}
    \end{table}

    Depending on the file type, jGnash may generate an intermediate file for the conversion process.
    When saving to a relational database, the process may take awhile to complete.

    \section{File | Export | Export Accounts and File | Import | Import Accounts}
    Use of these commands allows you to export your account structure and import it back into and new file.
    This is handy if you want to start a new file without manually recreating your accounts.
    This does not preserve existing transactions.

    \section{Change Database Password}
    By default, when a new relational database is created, a password is not specified. This allows you to password protect
    your file. This does not encrypt your data, so a person with the right tools can \underline{easily} access your data.
    It is useful for casual protection only. If encryption is important, use OS level encryption capability available on
    any modern operating system. This is disabled while a file is open.

    \section{Shutdown Server}
    Issues a shutdown request to a remote server.
    This is disabled while a file is open.


    \chapter{Plugins and JavaScript}
    jGnash support the addition of JavaScript and Plugins to add additional functionality to the application.

    \section{Plugins}
    Plugins are tightly integrated into jGnash, and once loaded, behave as if they are a standard part of the application.
    Plugins are coded in Java using a jGnash specific API as the entry point so they may be loaded into jGnash.

    Standard Plugins are packaged into \texttt{JAR} files and are typically located within the \directory{plugins}
    directory located in the directory jGnash is installed.

    You have two options for manually installing a Plugin:
    \begin{itemize}
        \item Place the JAR file into the \directory{plugins} sub directory where jGnash is installed and restart jGnash.
        \item Place the JAR file into your home directly where jGnash will find it.
        \begin{itemize}
            \item \directory{\$HOME/.jgnash/plugins} for UNIX and BSD based operating systems.
            \item \directory{USER\_HOME\textbackslash AppData\textbackslash Local\textbackslash jgnash\textbackslash plugins} for Windows operating systems.
        \end{itemize}
    \end{itemize}

    The jGnash JavaDoc may be referenced if you are interested in creating a jGnash Plugin.
    The MT940 import is written as a standard Plugin and may be referenced as an example of how to write one.

    \section{JavaScript}
    \label{sec:javascript}

    In addition to use of Plugins, jGnash allows you to create and run JavaScript programs.
    The internals of the jGnash engine and some user interface functions can be accessed to create custom reports, create
    and modify transactions, etc.

    Running a JavaScript program is as simple as using \menu{Tools > Run JavaScript} command from the menu bar.

    Below is an example JavaScript program that displays the accounts in the currently loaded file and demonstrates how to
    display a simple dialog.
    To try the program, create a text file using your favorite editor with a name of your choice that ends with
    a \texttt{.js} extension.
    After creating the file, simply using the \menu{Tools > Run JavaScript} command to select the program and run it.
    \\
    \begin{lstlisting}[language=JavaScript,numbers=left]
        load("nashorn:mozilla_compat.js"); // Load compatibility script

        importPackage(javax.swing);
        importPackage(Packages.jgnash.ui);
        importPackage(Packages.jgnash.engine);

        // helper function to print messages to the console
        function debug(message) {
            java.lang.System.out.println(message);
        }

        // show the console dialog to see the debug information
        var Console = Java.type("jgnash.uifx.views.main.ConsoleDialogController");
        Console.show();

        // this is how to get the default Engine instance
        var engine = EngineFactory.getEngine(EngineFactory.DEFAULT);

        // get a list of accounts
        var accountList = engine.getAccountList();

        // loop and print the account names to the console
        for (var i = 0; i; accountList.size(); i++)
        {
            var account = accountList.get(i);
            debug(account.getName());
        }
    \end{lstlisting}

    \notebox{
    JavaScript files are expected to be encoded as UTF-8 Files. An incorrect file encoding may cause script failures.
    }

    JavaScript programs have the advantage of not requiring the use of an IDE or Java compiler to create and test a program.
    The disadvantage is troubleshooting syntax and logic errors can be more difficult than writing a jGnash Plugin.

    % ======================
    % Command Line Options
    % ======================
    \chapter{Command Line Options}
    \label{ch:cmdOptions}

    jGnash has several command line options for advanced users.

    Parameters such as file names that include a space in the path must be escaped using double quotes.

    \begin{mdframed}[style=info]
        \textbf{Example} \\ \\
        The path to the file "/home/craig/jgnash files/jgnash.h2.db" must be escaped as shown.
        \\ \\
        \texttt{jGnash -{}-server "/home/craig/jgnash files/jgnash.h2.db" -{}-password fh56dy}
    \end{mdframed}

    \section{Options}
    \begin{description}
        \item[-h, -{}-help]
        Detailed display of all command line options.
        \item[-f, -{}-file \textit{filename}]
        Specifies a file to load at startup.
        \item[-p, -{}-portable]
        If portable is specified on the command line, jGnash preferences will be stored to a file name \texttt{pref.xml}
        instead of using the system registry.
        Use of this option is intended for users who want to run jGnash from a thumb drive on multiple computers and
        maintain their preferences without using the system registry.
        The \texttt{pref.xml} file will usually be stored at the location jGnash was started from.
        \item[-{}-portableFile \textit{filename}]
        If you don't like the location the \texttt{pref.xml} file is stored, or wish to use a different name, use
        this option to change the location and name to suit.
        \item[-{}-bypassBootloader]
        Bypasses the boot loader and requires manual installation of any OS specific files.
        \item[-u, -{}-uninstall]
        Removes all registry and configuration settings jGnash has created.
        This will not have any effect if you have been using the \texttt{--portable} option.
    \end{description}

    \section{Client/Server Options}
    \begin{description}
        \item[-{}-server \textit{filename}]
        Starts the jGnash server using the specified file which must be must be a jGnash relational database.
        The file must exist and not be in use by another program.
        A user interface will not be displayed.
        \item[-{}-host \textit{servername}]
        Specifies the name of the remote server.
        This starts jGnash and automatically connects to the specified server.
        If running on the same computer as the server, \texttt{localhost} may be used as the name of the server.
        \item[-{}-shutdown]
        Issues a shutdown request to a server.
        If -{}-host is not specified, then \texttt{localhost} is assumed for the server name
        \item[-{}-password \textit{password}]
        The password that the client must correctly specify to connect to the jGnash database.
        This is not required if the database is not protected.
        A password does nothing to encrypt a file.
        \item[-{}-port \textit{port}]
        An empty port for network communications.
        The specified port and \texttt{port+1} may not be used by any other application at the same time.
        The default port is 5300.
    \end{description}

    \notebox{
    It is possible to start the jGnash client and specify the server, and password settings from
    the \menu{File > Open} dialog.
    }
    \newpage
    \section{Client/Server Examples}
    Start the jGnash server using the default port with a password protected database
    \begin{mdframed}[style=info]
        \texttt{jGnash -{}-server "/home/craig/jgnash.mv.db" -{}-password fh56dy}
    \end{mdframed}

    Start the jGnash client and connect to the local server running a password protected database
    \begin{mdframed}[style=info]
        \texttt{jGnash -{}-host localhost -{}-password fh56dy}
    \end{mdframed}

    Issue a shutdown request to a remote server that is password protected
    \begin{mdframed}[style=info]
        \texttt{jGnash -{}-shutdown -{}-host serv1 -{}-password fh56dy}
    \end{mdframed}


    Issue a shutdown request to a local server that is not password protected
    \begin{mdframed}[style=info]
        \texttt{jGnash -{}-shutdown}
    \end{mdframed}

    \chapter{Frequently Asked Questions}\label{ch:frequently-asked-questions}

    \begin{description}
        \item[Where do I set my opening account balance?]
        Create an Equity Account called "Opening Balances" and transfer money from the Equity account to they
        new account which will establish an opening balance.
        The "Opening Balances" account may be hidden later if you do not want it visible.
        \item[What happened to transaction categories?]
        Most commercial personal finance applications use categories to help track spending and income.
        jGnash uses Income and Expense accounts instead of categories for tracking where your money
        comes from and where it goes.
        \item[Can I use multiple currencies?]
        Yes, the \menu{Tool > Currencies > Add/Remove} menu will let you add additional currencies.
        \\ \\
        After adding new currencies, simple create new accounts that use the new currency.
        When creating a transaction between accounts with different currencies, a field for the exchange rate will be enabled.
        \item[How do I add Securities / Stocks to my Investment and Mutual Fund Account?]
        First, you need to have created your stocks/securities. \menu{Tools > Commodities > Create / Modify}.
        \\ \\
        When creating the securities, the scale field must be filled in and the prefix field should be filled in.
        The scale will generally be the same scale as the currency the securities value is reported in.
        In most cases, a scale of 2 will work fine.
        For the prefix, the currency prefix of the reported value should be used.
        \\ \\
        After creating your securities, you can go back and modify the existing account or select the securities when
        creating a new account.
        Use the \menu{Securities} button in the dialog to make changes.
    \end{description}
    \begin{appendices}
        \chapter{Keyboard Shortcuts}\label{ch:keyboard-shortcuts}
        This Appendix contains application keyboard shortcuts that are available to you throughout jGnash

        Depending on your operating system and how you have it configured, other shortcuts may be
        available that perform the same function.
        
        \begin{table}[H]
            \begin{tabular}{|l|l|}
                \hline
                \textbf{Keys} & \textbf{Function} \\
                \hline
                \hline
                \keys{CTRL + F4}& Closes the active register window if you have one open \\
                \hline
                \keys{CTRL + A} & Displays the About dialog \textit{(When a register table does not have the focus)}\\
                \hline
                \keys{CTRL + A} & Selects all transactions \textit{(When a register table has the focus)}\\
                \hline
                \keys{CTRL + C} & Copies selected transaction information to the clipboard \\
                \hline
            \end{tabular}
            \caption{Shortcut Keys}
        \end{table}

        \begin{table}[H]
            \begin{tabular}{|l|l|}
                \hline
                \textbf{Keys} & \textbf{Function} \\
                \hline
                \hline
                \keys{CTRL + C} & Copy \\
                \hline
                \keys{CTRL + X} & Cut \\
                \hline
                \keys{CTRL + V} & Paste \\
                \hline
            \end{tabular}
            \caption{Editing Keys}
        \end{table}

        \include{gpl-3.0}

        \include{fdl-1.3}
        %\chapter{GNU Free Documentation License}

    \end{appendices}
\end{document}
